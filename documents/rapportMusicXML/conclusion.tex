\section{Conclusion}

Ce projet nous a permis de découvrir le domaine de la musique qui était obscur pour nous. De plus, nous ne pensions pas que l'informatique pouvait être utile à cet art.

\par
Nous avons aussi appliqué ce que nous avons appris lors du premier semestre en cours de développement d'un langage de programmation et en cours d'algorithmique. C'est à dire que pour passer d'un langage à un autre, il faut passer par un arbre.

\par
Comme nous l'avons exprimé précédemment, il existe des centaines de symboles musicaux. Tous les traiter dans le temps qui nous était imparti n'était pas raisonnable. Le programme a cependant été pensé pour permettre d'ajouter ces symboles en ajoutant seulement un condition à un \emph{If}. Nous aurions pu automatiser cela en parcourant une liste de symboles compatibles mais cela nous aurait obligé à faire des choix d'implémentation potentiellement regrettables.

\par
D'autre part, l'algorithme permettant de créer les groupes peut sans doute être simplifié. Nous pensons que certains cas traités sont peut être superflus. Nous aurions aussi pu lier les groupes et mesures de façon plus profonde mais nous ne pensons pas que cela aurait été d'une grande utilité car cette partie est assez généralisable pour pouvoir être rendu générique.
