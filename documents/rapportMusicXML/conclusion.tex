\section{Conclusion}

Ce projet nous a permis de découvrir le domaine de la musique qui était obscur pour nous. De plus, nous ne pensions pas que l'informatique pouvait être utile à cet art.

\par
Comme nous l'avons éxprimé précedemment, il existe des centaines de symboles musicaux. Tous les traiter dans le temps qui nous était impartie n'était pas résonable. Le programme à cependant été pensé pour permettre d'ajouter ces symboles en ajoutant seulement un condition à un \emph{If}. Nous aurions pu automatiser cela en parcourrant une liste de symbole compatible mais cela nous aurai obligé à faire des choix d'implémentation potentiellement regrettables.

\par
D'autre part, l'algorithme permettant de créer les groupes peut sans doute être simplifié. Nous pensons que certains cas traité son peut être superflu. Mais jusqu'à présent il fonctionne, il n'était pas dans nos priorités. Nous aurions aussi pu lié les groupes et mesures de façon plus profonde mais nous ne pensons pas que cela aurai été d'une grande utilité car cette partie assez généralisable pour pouvoir être rendu générique.
