\section{Architecture du programme}
\par
Dans cette section nous aborderons l'architecture du programme. C'est-à-dire
la façon dont est organisé notre code et notamment les choix d'implementation
que nous avons effectués. Pour rappel, notre objectif est de générer un arbre rythique
et de récupérer la liste des accords et symboles musicaux pour réaliser l'affichage
de la partition musical.

\subsection{Architecture general}
\par
Le programme se divise en trois parties. La première étant le parser. Cette partie
permet au programme de disposer d'un \emph{DOM} à partir d'un fichier MusicXML.
La seconde partie est la représentation objet des données contenues dans le \emph{DOM}.
Et enfin la troisième partie est l'arbre rythmique qui pourra être généré à partir
de la représentation objet de la partition.

\subsection{Parsing du fichier MusicXML}
\par
La première étape est de parser le document afin d'extraire les informations qu'il
contient. C'est là qu'interviennent les éléments contenus dans le package
\emph{pstl.musicxml.parsing}. Nous avons décidé d'encapsuler le parser XML natif
de Java dans une class \emph{XMLParser}. Ce choix est motivé pour plusieurs raisons.
\par
Tout d'abord pour ne laisser apparent que les fonctions du parser de Java que
nous allons réellement utiliser afin de limiter le risque que nous ou un utilisateur
tiers fasse usage du parser d'une mauvaise façon. Nous avons aussi fait ce choix pour simplifier
l'utilisation de la class pour que la création du parser qu'elle encapsule se fasse de la même façon à chaque fois.
 Et enfin ce choix a été rendu nécessaire à cause de \emph{Relax NG}. En effet comme nous
ne faisons appel ni aux fichiers \emph{DTD} ni aux fichiers \emph{XML Schema}, il nous
faut effectuer un prétraitement pour éliminer les références aux \emph{DTD} du
fichier à parser. Vous pourriez vous demander pourquoi ne pas simplement
utiliser les \emph{DTD} pour valider le document ? La raison est simple,
la plupart des fichiers font référence aux \emph{DTD} en ligne fournis par
MusicXML. Or ces derniers n'autorisent que les navigateurs web à accéder à de tels
fichiers. Les possibilitées qui s'offraient à nous étaient les suivantes : se faire
passer pour un navigateur en modifiant quelques variables d'environement. Cela
aurait eu pour désavantage tout d'abord de ne pas être très rapide, l'accès à
aux \emph{DTD} par réseau n'est pas très rapide comparé à un accès local.
Nous jugions d'autre part la méthode peu honnête. En effet si l'organisation derrière MusicXML
ne permet pas cela pour des raisons que j'imagine financières (par cela j'entends le coût engendré par la maintenance des serveurs)
 il n'aurait pas été juste d'outrepasser leurs instructions. Et enfin le dernier choix non retenu était celui de faire appel
à des \emph{DTD} stockées localement. Cette méthode a été écartée bien que conseillée
dans la documentation du format MusicXML car elle impliquait des redirections d'URI
ce qui aurait fortement complexifié la création du parser.
\par %TODO Mettre reference vers le git de la grammaire rng de music xml
C'est donc pour toutes ces raisons que nous avons choisi d'utiliser \emph{Relax NG}.
De plus, les fichiers décrivants la grammaire de MusicXML sont disponibles en ligne
et l'auteur a fait preuve d'une certaine exhaustivité lors de la rédaction de ces derniers.
\par
Ce parser nous permet de disposer d'un \emph{DOM} (qui a pour nom de class \emph{Document} avec Java)
qui pourra être parcouru plus tard.

\subsection{Représentation objet de la partition}
\par
Nous aurions pu nous satisfaire du \emph{Document} retourné par le parser pour
générer notre arbre rythmique mais cela aurait posé plusieurs problèmes. Tout d'abord
la manipulation du \emph{DOM} n'est pas aisée. Cette structure de données basé sur
des noeuds qui contiennent des fils, attributs et leur contenu textuel, doit
être exploré d'une façon assez lourde à cause en partie du fait qu'un noeud n'est
pas nativement un objet iterable comme les \emph{Collections} de Java par exemple.
D'autre part, toutes les données contenues dans ces noeuds sont considérées comme des chaines
de caractère qui nécessitent un parsing et donc une manipulation assez verbeuse.
De ce fait, il est plus simple pour nous de parcourir ce \emph{Document} une seule fois et d'extraire les données qu'il
contient dans une structure de données plus aisément manipulable dans java. Cela
permet par exemple d'éviter des erreurs lors du développement des autres fonctionnalités de
l'application qui se basent sur ces informations.

%Mettre des figures pour expliciter les partitions ? Un diagramme de class par exemple ?
%TODO établir un vrai package pour les partitions.
\par
La structure de données que nous avons élaborée se compose de la façon suivante :
l'élément qui va contenir toutes les informations est la class \emph{Score}. Une
\emph{Score} contient une liste de \emph{Parts} qui l'on peut qualifier de
\emph{Voix} en français. Chaque \emph{Part} contient un nom, un identifiant ainsi qu'une liste
de \emph{Measures} (mesure) qui contient elle-même un numéro, une \emph{Signature} et une
liste de \emph{Chord} (accord). Un accord est composé d'une liste d'éléments musicaux
appelés \emph{MusicalItems} dans le programme. Chaque élément peut être soit une note
un silence ou une liaison. Un élément musical possède aussi une liste de symboles
musicaux. Un symbole musical représente une notation qui peut être une nuance comme
un point ou encore un répétition par exemple.
\par
On pourrait penser ce modèle de données lourd, mais cela est largement compensé entre autres
par l'aisance d'utilisation qu'il procure ainsi qui la possibilité notamment de déduire les
"coordonnées" des éléments qui le composent. En effet rien de plus facile que de dire
qu'une note se trouve dans le \emph{Chord} 4 de la \emph{Measure} 2 de la \emph{Part}
2 de telle \emph{Score}. Ainsi nous pouvons par exemple utiliser ces coordonnées pour
placer certains symboles à des endroits précis.
