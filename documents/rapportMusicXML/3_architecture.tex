\section{Architecture du programme}

\par
Dans cette section nous aborderons l'architecture du programme. C'est-à-dire
la façon dont est organisé notre code et notamment les choix d’implémentation
que nous avons effectués. Pour rappel, notre objectif est de générer un arbre rythmique
et de récupérer la liste des accords et symboles musicaux pour réaliser l'affichage
de la partition musical.

\subsection{Architecture générale}

\par
Le programme se divise en trois parties. La première étant le parser. Cette partie
permet au programme de disposer d'un \emph{DOM} à partir d'un fichier MusicXML.
La seconde partie est la représentation objet des données contenues dans le \emph{DOM}.
Et enfin la troisième partie est l'arbre rythmique qui pourra être généré à partir
de la représentation objet de la partition.

\subsection{Parsing du fichier MusicXML}

\par
La première étape est de parser le document afin d'extraire les informations qu'il
contient. C'est là qu'interviennent les éléments contenus dans le package
\emph{pstl.musicxml.parsing}. Nous avons décidé d'encapsuler le parser XML natif
de Java dans une classe \emph{XMLParser}. Ce choix est motivé pour plusieurs raisons.

\par
Tout d'abord pour ne laisser apparent que les fonctions du parser de Java que
nous allons réellement utiliser afin de limiter le risque que nous ou un utilisateur
tiers fasse usage du parser d'une mauvaise façon. Nous avons aussi fait ce choix pour simplifier
l'utilisation de la classe pour que la création du parser qu'elle encapsule se fasse de la même façon à chaque fois.
Et enfin ce choix a été rendu nécessaire à cause de \emph{Relax NG}. En effet comme nous
ne faisons appel ni aux fichiers \emph{DTD} ni aux fichiers \emph{XML Schema}, il nous
faut effectuer un prétraitement pour éliminer les références aux \emph{DTD} du
fichier à parser. Vous pourriez vous demander pourquoi ne pas simplement
utiliser les \emph{DTD} pour valider le document ? La raison est simple,
la plupart des fichiers font référence aux \emph{DTD} en ligne fournis par
MusicXML. Or ces derniers n'autorisent que les navigateurs web à accéder à de tels
fichiers. Les possibilités qui s'offraient à nous étaient les suivantes : se faire
passer pour un navigateur en modifiant quelques variables d’environnement. Cela
aurait eu pour désavantage tout d'abord de ne pas être très rapide, l'accès à
aux \emph{DTD} par réseau n'est pas très rapide comparé à un accès local.
Nous jugions d'autre part la méthode peu honnête. En effet si l'organisation derrière MusicXML
ne permet pas cela pour des raisons que j'imagine financières (par cela j'entends le coût engendré par la maintenance des serveurs)
il n'aurait pas été juste d'outrepasser leurs instructions. Et enfin le dernier choix non retenu était celui de faire appel
à des \emph{DTD} stockées localement. Cette méthode a été écartée bien que conseillée
dans la documentation du format MusicXML car elle impliquait des redirections d'URI
ce qui aurait fortement complexifié la création du parser.


\par
C'est donc pour toutes ces raisons que nous avons choisi d'utiliser \emph{Relax NG}.
De plus, les fichiers décrivant la grammaire de MusicXML sont disponibles en ligne
et l'auteur, qui les a déposés sur un projet GitHub \cite{relaxng_for_musicxml},
a fait preuve d'une certaine exhaustivité lors de la rédaction de ces derniers.

\par
Ce parser nous permet de disposer d'un \emph{DOM} (qui a pour nom de classe \emph{Document} avec Java)
qui pourra être parcouru plus tard. Le schéma suivant récapitule les étapes pour passer du document XML au DOM.


\begin{figure}[!h]
\centering
\includegraphics[width=0.5\textwidth]{parsing_xml_to_dom.png}\\[1cm]
%source : http://www.amfastech.com/2014/11/complete-tutorial-on-selecting-best-xml-parser.html
\caption{Parsing d'un document XML en DOM}
\label{Parsing d'un document XML en DOM}
\end{figure}


\subsection{Représentation objet de la partition}

\par
Nous aurions pu nous satisfaire du \emph{Document} retourné par le parser pour
générer notre arbre rythmique mais cela aurait posé plusieurs problèmes. Tout d'abord
la manipulation du \emph{DOM} n'est pas aisée. Cette structure de données basé sur
des nœuds qui contiennent des fils, attributs et leur contenu textuel, doit
être exploré d'une façon assez lourde à cause en partie du fait qu'un nœud n'est
pas nativement un objet itérable comme les \emph{Collections} de Java par exemple.
D'autre part, toutes les données contenues dans ces nœuds sont considérées comme des chaînes
de caractère qui nécessitent un parsing et donc une manipulation assez verbeuse.
De ce fait, il est plus simple pour nous de parcourir ce \emph{Document} une seule fois et d'extraire les données qu'il
contient dans une structure de données plus aisément manipulable dans Java. Cela
permet par exemple d'éviter des erreurs lors du développement des autres fonctionnalités de
l'application qui se basent sur ces informations. Nous utilisons donc un ensemble de méthodes
contenus dans la classe \emph{ScoreUtils} pour convertir notre \emph{Document} en instance de \emph{Score}.

\par
\emph{ScoreUtils} intègre aussi une partie très importante, la récupération des symboles contenus dans la partition.
En effet, une partition n'est pas seulement un ensemble de notes il contient aussi un grand nombre de symbole ayant
tous des significations très differentes pourvant aussi bien influancer la tonalité de la note ainsi que sa durée.

%Mettre des figures pour expliciter les partitions ? Un diagramme de class par exemple ?
%TODO établir un vrai package pour les partitions.

\par
La structure de données que nous avons élaborée se compose de la façon suivante :
l'élément qui va contenir toutes les informations est la classe \emph{Score}. Une
\emph{Score} contient une liste de \emph{Parts} qui l'on peut qualifier de
\emph{Voix} en français. Chaque \emph{Part} contient un nom, un identifiant ainsi qu'une liste
de \emph{Measures} (mesure) qui contient elle-même un numéro, une \emph{Signature}, une
liste de \emph{IMusicalItem} et un \emph{Metronome} correspondant au tempo.

\par
\emph{IMusicalItem} est la super classe de bon nombre d'élément que nous pouvons qualifier
de bout de chaine comme les \emph{Note}, \emph{Rest} (silence) ou encore les \emph{Tie} (liaison).
Mais ce n'est pas tout, les groupes de \emph{IMusicalItem} sont également des \emph{IMusicalItem}.
Cela nous permet entre autre de pouvoir imbriquer des groupes de \emph{IMusicalItem}.
Dans les premières version de cette partie du code, il y n'y avait pas de lien entre les
mesures et les groupes. Nous nous sommes rapidement rendu compte que la mesure et le groupe étaient
a peu de chose près la même chose. Nous avons donc décidé qu'il serait mieux que le mesure soit
aussi un groupe pour limiter la duplication de code. Un groupe peut représenter des croches liées ou
encore un \emph{triolet}.

\par
Le tempo contient un type (une noire par exemple) et un nombre par minute. Ainsi, le tempo \emph{60 à la noire} signifie qu'il y a,
dans une minute, 60 noires ou bien 30 blanches. Un élément musical possède aussi une liste de symboles
musicaux. Un symbole musical représente une notation qui peut être une nuance comme
un point ou encore un répétition par exemple.

\par
Pour chaque symbole musical, une classe lui est associée. Le symbole peut être unaire ou binaire.
Les symboles unaires sont liés à une seule note, tel que les points d'orgues. Les symboles binaires
sont liés à 2 notes, comme les liaisons. Pour chaque note, on récupère tous les symboles qui lui
sont associés, on créé les objets intermédiaires et on les ajoute à l'objet \emph{Note}.

\par
La création de l'objet \emph{Score} se fait en deux étapes. Tout d'abord nous récuperons
les données brutes dans le \emph{Document} que nous mettons dans l'objet \emph{Score} final.
Une fois toutes ces données à disposition nous pouvons créer les groupes en fonction des
symboles que nous rencontrons lors du parcours des mesures. Prenons comme exemple la
balise \emph{<beam>} qui peut être présente dans une balise \emph{<note>} et qui
contient les informations nécessaires pour représenter un groupe de
croches dans \emph{MusicXML}. Cette balise a principalement un attribut \emph{number} qui designe
le numéro du groupe actuel dans le cas où il a plusieurs groupes dans une mesure et
qui contient l'énumeration suivante : \emph{begin, end, continue} qui nous permet de
connaitre la position de la note dans le groupe. Lors de notre premier passage nous
stockons ces informations dans un objet \emph{Beam}. Lors de notre second passage
nous utilisons ces données pour créer nos groupes.
% TODO mettre ref à la doc de MusicXML ->
% http://usermanuals.musicxml.com/MusicXML/MusicXML.htm#EL-MusicXML-beam.htm%3FTocPath%3DMusicXML%2520Reference%7CScore%2520Schema%2520(XSD)%7CElements%7Cnote%7C_____18

\par
On pourrait penser que ce modèle de données est lourd, mais cela est largement compensé entre autres
par l'aisance d'utilisation qu'il procure ainsi que la possibilité notamment de déduire les
"coordonnées" des éléments qui le composent. En effet rien de plus facile que de dire
qu'une note se trouve dans le \emph{Chord} 4 de la \emph{Measure} 2 de la \emph{Part}
2 de telle \emph{Score}. Ainsi nous pouvons par exemple utiliser ces coordonnées pour
placer certains symboles à des endroits précis.


\subsection{Construction des arbres rythmiques}

Une fois la structure objet construite, les arbres rythmiques pour chaque mesure sont construits.
Cette étape est réalisée par les classes du package \emph{rhythmicstructures}. La classe factory
\emph{RhythmicTreeFactory} parcours tout l'arbre des objets intermédiaires, et pour chaque \emph{Measure}
rencontrée, un nouvel objet de la classe \emph{RhythmicTree} est créé. Ce dernier contient une
\emph{Signature}, une \emph{Fraction}, un \emph{ItemType}, une liste d'\emph{ExtraSymbols} et une
liste de \emph{RhythmicTree} correspondant aux fils.


\subsection{Test du programme}

Afin de vérifier que notre programme fonctionne correctement, une base de test a été créée dans le dossier \emph{test-data}. Elle est constituée de nombreux fichiers tests au format MusicXML. Chaque fichier contient quelques notes avec un symbole musical (signe crescendo, staccato, etc.). Chaque classe associée à un symbole peut ainsi être testée. Une partition complète de Bach est également présente pour tester sur un grand nombre de mesure.
