\section{Autre section}

\subsection{Parsing d'un fichier MusicXML}

Un document XML est chargé puis vérifié grâce à la grammaire de MusicXML. Si le document est conforme, il est parsé en DOM ce qui permet de travailler dessus facilement.

\begin{figure}[!h]
\centering
\includegraphics[width=0.5\textwidth]{parsing_xml_to_dom.png}\\[1cm]
%source : http://www.amfastech.com/2014/11/complete-tutorial-on-selecting-best-xml-parser.html
\caption{Parsing d'un document XML en DOM}
\label{Parsing d'un document XML en DOM}
\end{figure}



\subsection{Composition d'un morceau}


Un morceau est composé de plusieurs portées et/ou plusieurs systèmes de portée.
\par
Un système de portée est constitué de portées.
\par
Une porté est liée à un instrument et est composée d’une clé, d’une armure, d’un chiffrage de la mesure et de mesure.
\par
Une clé contient un nom et un numéro de ligne.
\par
Une armure est composée d'un nombre entier relatif. S'il est négatif alors l'armure contient des bémols, s'il est positif alors elle contient des dièses, s'il vaut 0 alors rien n'est affiché.
\par
Le chiffrage de la mesure contient un numérateur, représentant le nombre de temps de la mesure, et dénominateur, représentant la durée d’une mesure.
\par
Une mesure est composée d'une clé optionnelle, d'un chiffrage de la mesure optionnel, de notes, de notes reliées par une division artificielle du temps (duolet/triolet/quartolet/...), d'accords, de silences, de notes reliées par une liaison, de signes indiquant si la mesure est répétée et comment et de notes jouées en crescendo / decrescendo.
\par
Une note est constituée d'un nom, d'une série, d'une durée, d'une proportion par rapport à la ronde, d'un point indiquant si elle est pointée ou pas, d'un volume, de paroles et d'une altération (bémol, dièse ou remise à la normal).
\par
Un silence est composé d'un nom, d'une série, d'une durée, d'une proportion par rapport à la ronde et d'un point indiquant si elle est pointée ou pas.

%Récapitulatif sous forme de liste:\\
%\begin{itemize}
%\item morceau : plusieurs portées et/ou plusieurs systèmes de portée
%\item système de portée : portées
%\item une porté est lié à un instrument et est composée d’une clé, d’une armure, d’un chiffrage de la mesure et de mesure
%\item clé : nom et numéro de ligne
%\item armure : nombre entier relatif, négatif : bémol, positif : dièse
%\item chiffrage de la mesure : numérateur représentant le nombre de temps de la mesure et dénominateur représentant la durée d’une mesure
%\item mesure : clé optionnelle, chiffrage de la mesure optionnel, notes, notes reliées par une division artificielle du temps (duolet/triolet/quartolet/...), accords, silences, notes reliées par une liaison, si la mesure est répétée et comment, notes jouées en crescendo / decrescendo
%\item note : nom, série, durée, proportion par rapport à la ronde, pointée ou pas, volume, paroles, altération (bémol, dièse ou normal), tiret
%\item silence : nom, série, durée, proportion par rapport à la ronde, pointée ou pas
%\end{itemize}