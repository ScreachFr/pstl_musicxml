\section{Technologies utilisées}

\subsection{OpenMusic}

OpenMusic \cite{openmusic} est un langage de programmation visuel basé sur le langage Lisp qui permet d'écrire graphiquement des compositions musicales. Il a été conçu par les chercheurs de l'IRCAM Carlos Agon, Gérard Assayag et Jean Bresson. Les programmes sont constitués d'éléments reliés entre eux et représentant des structures de données ou des fonctions. La figure suivante montre l'interface du logiciel.


\begin{figure}[!h] %h : here
\centering
\includegraphics[width=1\textwidth]{openmusic.png}\\[1cm]
\caption{OpenMusic}
\label{OpenMusic}
\end{figure}

\par
Dans ce logiciel, les morceaux de musique sont constitués d'une suite d'arbres rythmiques. Comme décrit dans l'article \cite{agon}, \enquote{un arbre rythmique est défini comme un couple (D S) où D est une fraction (< 0) et S est une liste de n-éléments définissant n-proportions de D. Chaque élément de S peut-être soit un entier, soit un arbre rythmique.}

\par
Les images suivantes sont des exemples d'arbres rythmiques, en haut, et la partition correspondante, en bas.


\begin{minipage}[c]{.46\linewidth}
  \centering
  \includegraphics[width=0.5\textwidth]{rt1.png}\\[1cm]
\end{minipage}
\hfill
\begin{minipage}[c]{.46\linewidth}
  \centering
  \includegraphics[width=0.9\textwidth]{rt2.png}\\[1cm]
\end{minipage}



\subsection{Le langage Java}

\par 
Java \cite{java_oracle} est un langage de programmation \textbf{orienté objet} fortement typé développé par \textbf{Sun Microsystems} à partir de \textbf{1995}. La société sera plus tard rachetée par \textbf{Oracle} en 2009 qui possède et maintient Java encore aujourd'hui.

\par 
Java se détache de la masse des autres langages de programmation notamment grâce à sa portabilité et sa facilité d'utilisation.

\begin{lstlisting}[caption=Hello world en java]
public class HelloWorld {
    public static void main(String[] args) {
        System.out.println("Hello world!");
    }
}
\end{lstlisting}

\par
Ci-dessus, un classique "Hello world" en Java. Nous pouvons y voir la définition de la classe \emph{HelloWorld} ainsi que la méthode principale du programme nommé \emph{main} et enfin un affichage sur la sortie standard.



\subsection{Le langage XML}

Le langage XML \cite{xml_w3c, xml_ocr}, acronyme de e\textbf{X}tensible \textbf{M}arkup \textbf{L}angage, est langage de balisage générique spécifié par le W3C. Il permet de définir différents espaces de noms, c'est à dire des langages avec leur propre grammaire et vocabulaire. Il permet l'échange d'information entre des programmes très différents à condition d'utiliser la même grammaire.

\par
Il a l'avantage de pouvoir être compris par les êtres humains et les machines. Cependant, c'est un langage verbeux et qui peut donc prendre beaucoup de place s'il contient beaucoup d'information.

\par
Un document XML est constitué de balises pouvant contenir d'autres balises ou une valeur simple. Une balise peut aussi contenir des attributs donnant des informations supplémentaires sur le contenu.

\begin{lstlisting}[caption=Exemple d'un document XML]
<?xml version="1.0" encoding="UTF-8"?>
<racine>
    <balise attribut="valeur">Contenu</balise>
    <baliseunique />
</racine>
\end{lstlisting}

\par
Ci-dessus un exemple de XML simpliste mais qui met en avant les bases du langage. La première ligne annonce le type de document et la version dans lequel le document va être rédigé. \emph{<racine>} est le nœud racine du document, celui qui en somme va contenir tout le document. On peut ici, facilement remarquer que le document peut être représenté sous la forme d'un arbre.

\par
XML permet à l'utilisateur de définir lui même la grammaire de son document grâce notamment aux \textbf{DTD} et au \textbf{Schéma XML}. Ces outils nous permettent de disposer de format d’échange de données tel que \textbf{MusicXML}.



\subsection{Le format MusicXML}

MusicXML \cite{musicxml} est un format de fichier permettant de représenter la notation musicale occidentale (notation classique, accords en notation anglo-saxonne, tablatures et percussions) et basé sur le langage XML. Il est propriétaire mais il peut librement utilisé avec une licence publique.

\par
Il y a plus de 20 ans, le format MIDI était très utilisé. Cependant, il n’est pas très adapté pour représenter toutes les caractéristiques de la musique, on perd donc en informations avec ce format. Pour pallier à cela, les formats SMDL et NIFF ont été créés. Cependant, le format SMDL était complexe et donc peu compréhensible. Il était donc très peu utilisé. Le format NIFF était un format peu pratique à utiliser et n’a donc pas été adopté par certains logiciels. Ces formats n’ont donc pas eu le succès souhaité.

\par
En 2004, la société Recordare LLC s’inspire des 2 formats universitaires MuseData et Humdrum pour créer la première version du format MusicXML. Ses avantages sont qu’il est facile à manipuler. Il permet le transfert de morceaux de musique d’une application à une autre. Il peut représenter beaucoup de caractéristiques de la musique. Cependant, il est verbeux, puisqu'il utilise le format XML, et ne donc permet pas de représenter la musique non occidentale.

\par
Il est de plus en plus utilisé puisque plus de 200 logiciels de musique l’ont adopté. Il est donc possible de travailler finement sur un morceau de musique en utilisant différents programmes.

\par
Comme le format XML est verbeux, le fichier prend de la place. La version 2.0, sortie en 2007, apporte donc la compression du fichier au format xml en un fichier au format mxl, et permet de diviser sa taille de façon importante. La version 3.0, sortie en 2011, permet le support des instruments virtuels.\\~\\

\par
On voit, dans le code correspondant à la partition suivante, que les informations sur la partition sont placées dans la balise "measure" et celle concernant la ronde sont contenues dans la balise "note".

\begin{figure}[!h] %h : here
\centering
\includegraphics[width=0.2\textwidth]{musicxml_hello_world.png}\\[1cm]
%source : https://en.wikipedia.org/wiki/MusicXML
\caption{Hello World en MusicXML}
\label{Hello World en MusicXML}
\end{figure}


\begin{lstlisting}[caption=Document XML d'un Hello World en MusicXML, label=ruleml]
<?xml version="1.0" encoding="UTF-8" standalone="no"?>
<!DOCTYPE score-partwise PUBLIC
    "-//Recordare//DTD MusicXML 3.0 Partwise//EN"
    "http://www.musicxml.org/dtds/partwise.dtd">
<score-partwise version="3.0">
  <part-list>
    <score-part id="P1">
      <part-name>Music</part-name>
    </score-part>
  </part-list>
  <part id="P1">
    <measure number="1">
      <attributes>
        <divisions>1</divisions>
        <key>
          <fifths>0</fifths>
        </key>
        <time>
          <beats>4</beats>
          <beat-type>4</beat-type>
        </time>
        <clef>
          <sign>G</sign>
          <line>2</line>
        </clef>
      </attributes>
      <note>
        <pitch>
          <step>C</step>
          <octave>4</octave>
        </pitch>
        <duration>4</duration>
        <type>whole</type>
      </note>
    </measure>
  </part>
</score-partwise>
\end{lstlisting}


\subsection{Le langage Relax NG}

\textbf{Relax NG} (\textbf{Re}gular \textbf{La}nguage for \textbf{X}ML \textbf{N}ext \textbf{G}eneration) \cite{relaxng} née de la fusion de TreX de James Clark et de Relax de Murata Makoto. C'est un langage qui permet de définir la grammaire d'un document XML. Relax NG ne s'intéresse qu'à la structure du document et non à sa valeur.

\par
C'est ce que nous utiliserons afin de s'assurer de la validité du document à traiter.


\subsection{Les librairies}

Dans cette section, nous aborderons les librairies utilisées pour réaliser ce projet. Cela ira de la validation du XML en passant par le parsing de ce dernier, jusqu'à la récupération des information stockées dans le fichier MusicXML.


\subsubsection{Trang et Jing}

\textbf{Trang} \cite{trang} et \textbf{Jing} \cite{jing} sont deux librairies développées par \textbf{Thai Open Source}. Elles permettant de générer des grammaires Relax NG et de valider des documents XML à partir de cette même grammaire.

\par
Trang est une librairie qui permet de traduire un fichier de description grammaticale en fichier Relax NG. En effet XML n'est pas facilement lisible pour un esprit humain, c'est pour cela que Trang nous permet de créer notre grammaire dans un langage plus compréhensible. Une fois la grammaire écrite dans un fichier en \emph{.rnc}, nous pouvons générer notre fichier Relax NG en \emph{.rng}.

\par
Jing, quant à lui, est une librairie Java qui permet de valider un document XML à l'aide d'un fichier Relax NG.


\begin{lstlisting}[caption=Code java permettant de vérifier la validation d'un document XML]

final ValidationDriver vd = new ValidationDriver();
vd.loadSchema(rngFile);

if (!vd.validate(inputTextStream)) {
	throw new ParseException("Invalid xml :(");
} else {
    System.out.println("Valid xml :)");
}
\end{lstlisting}

\par
Le code ci-dessus est une utilisation simplifiée de Jing. Nous commençons tout d'abord par créer un objet \emph{ValidationDriver} de Jing dans lequel nous chargeons notre fichier Relax NG. Nous n'avons ensuite plus qu'à lancer la méthode \emph{validate(InputSource in) : boolean} qui nous indiquera si le document est valide.


\subsubsection{L'API SAX}

SAX \cite{sax_website, sax_oracle} est une API créée par David Megginson en 1998 et est l'acronyme de \textbf{S}imple \textbf{A}PI for \textbf{X}ML. Elle permet de manipuler des documents XML en utilisant des événements envoyées à chaque rencontre d'un élément.


\subsubsection{Le DOM}

Le DOM \cite{dom_w3c} , ou \textbf{D}ocument \textbf{O}bject \textbf{M}odel, est une interface de programmation normalisée par le W3C et est indépendant de tout plateforme et langage. Il voit les documents à balises comme des arbres dont le contenu et la structure peuvent être accédés et mis à jour dynamiquement.

\par
La figure suivante montre un exemple de DOM.

\begin{figure}[!h]
\centering
\includegraphics[width=0.8\textwidth]{dom.png}\\[1cm]
\caption{Exemple d'un DOM d'une page HTML}
\end{figure}


\subsubsection{Affichage d'un graphe avec GraphStream}

Une fois le fichier XML parsé, nous disposons d'un \emph{DOM} sous forme d'un objet \emph{document}. Nous pouvons ensuite afficher l'arbre grâce à la librairie GraphStream \cite{graphstream}. Pour cela nous appelons un méthode qui affiche un nœud de l'arbre sur la racine du \emph{Document}. Cela nous permet donc d'afficher tout l'arbre.
