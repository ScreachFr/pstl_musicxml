\section{Introduction}


L'IRCAM \cite{ircam}, Institut de Recherche et Coordination en Acoustique/Musique, est un centre de création et de recherche scientifique sur la musique. Il a été fondé en 1969 par Pierre Boulez à la demande du président Georges Pompidou. 

\par
En 1995, le CNRS et le ministère de la Culture et de la Communication s'associe et créée l'UMR 9912 STMS. Cette unité mixte de recherche, hébergée à l'IRCAM, s'intéressent aux sciences et aux technologies de la musique et du son. En 2010, elle est rejoint par l'UPMC.

\par
Cette UMR est composé de nombreuses équipes de recherche. L'une d'elles s'intitule "Représentations Musicales", et réalise des outils de compositions musicales. Elle a notamment créée OpenMusic \cite{openmusic}, un environnement de composition musicale assistée par ordinateur.

\par
Les chercheurs de l'IRCAM développent actuellement un logiciel d'analyse musical se basant sur des partitions. Pour faciliter les échangent, ils souhaitent disposer de ressources leur permettant de lire des fichiers au format MusicXML. % a vérifier

%Les chercheurs de l'IRCAM souhaitent travailler sur des partitions musicales sous formes d'arbres rythmiques. Actuellement, ils utilisent OpenMusic. Cependant, ce logiciel --- (mettre inconvénients) ---. C'est pourquoi, ils souhaitent une nouvelle version. 

\par
Ce programme sera écrit en langage Java. Il permettra donc d'éditer graphiquement des structures de données musicales à partir d'un fichier. Son implémentation comportera différents modules, dont celui consistant à construire les arbres rythmiques à partir d'un fichier au format MusicXML. C'est ce module que notre tuteur de projet Carlos Agon, enseignant-chercheur à l'UPMC et membre de l'équipe de représentations musicales, nous a demandé de réaliser.

\par
Le module que nous avons développé est déposé sur un compte GitHub \cite{github_pstl} public. Il est donc open source. Il a été réalisé en anglais pour favoriser son internationalisation.

