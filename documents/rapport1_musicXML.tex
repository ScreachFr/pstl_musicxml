\documentclass[a4paper, 12pt]{article}

\usepackage[francais]{babel}
\usepackage[utf8]{inputenc} 
\usepackage[T1]{fontenc}

\title{Présentation du format MusicXML}
\author{Sébastien Duchenne et Alexandre Gaspard Cilia}
\date{15/01/2017}

\begin{document}

\maketitle

\par
MusicXML est un format de fichier permettant de représenter la notation musicale occidentale (notation classique, accords en notation anglo-saxonne, tablatures et percussions). Il est basé sur le langage XML et est libre de droit afin qu’il se répande.

\par
Ils existent beaucoup de programmes permettant de travailler sur les morceaux de musique. Il y a plus de 20 ans, le format MIDI était très utilisé. Cependant, il n’est pas très adapté pour représenter toutes les caractéristiques de la musique, on perd donc en informations avec ce format. 

\par
Pour pallier à cela, les formats SMDL et NIFF ont été créés. Cependant, le format SMDL était complexe et donc peu compréhensible. Il était donc très peu utilisé. Le format NIFF était un format peu pratique à utiliser et n’a donc pas été adpoté par certains logiciels. Ces formats n’ont donc pas eu le succès souhaité.

\par
En 2004, la société Recordare LLC s’inspire des 2 formats universitaires MuseData et Humdrum pour créer le format MusicXML. Ses avantages sont qu’il est facile à manipuler. Il permet le transfert de morceaux de musique d’une application à une autre. Il peut représenter beaucoup de caractéristiques de la musique. Cependant, il est verbeux et ne permet pas de représenter la musique non occidentale.

\par
Il est de plus en plus utilisé puisque 210 logiciels de musique l’ont adopté. Il est donc possible de travailler finement sur un morceau de musique en utilisant différents programmes. 

\par
Comme le format XML est verbeux, le fichier prend de la place. La version 2.0 apporte donc la compression du fichier .xml en un fichier .mxl. La version 3.0 apporte le support des instruments virtuels.


\end{document}
